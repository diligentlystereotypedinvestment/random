\usepackage{mathrsfs}
% \usepackage[usenames,dvipsnames]{color}
\usepackage{url}
\usepackage[utf8]{inputenc}
\usepackage[T1]{fontenc}
\usepackage{textcomp}
\usepackage{amsmath,amssymb,amsthm,mathtools}
\usepackage{bm}
% \usepackage[all,arc,2cell]{xy}
% \UseAllTwocells
\usepackage{enumerate}
\usepackage{graphicx}
% \usepackage[svgnames]{xcolor}

% %%% hyperref stuff is taken from AGT style file
\usepackage{hyperref,cleveref}
\hypersetup{%
  bookmarksnumbered=true,%
  bookmarks=true,%
  colorlinks=true,%
  linkcolor=blue,%
  citecolor=blue,%
  filecolor=blue,%
  menucolor=blue,%
  pagecolor=blue,%
  urlcolor=blue,%
  pdfnewwindow=true,%
  pdfstartview=FitBH}   

\let\fullref\autoref
\def\makeautorefname#1#2{\expandafter\def\csname#1autorefname\endcsname{#2}}
%
%  Some standard autorefnames.  If the environment name for an autoref
%  you need is not listed below, add a similar line to your TeX file:
%
%\makeautorefname{equation}{Equation}%
\def\equationautorefname~#1\null{(#1)\null}
% \makeautorefname{footnote}{footnote}%
\makeautorefname{item}{item}%
\makeautorefname{figure}{Figure}%
\makeautorefname{table}{Table}%
\makeautorefname{part}{Part}%
\makeautorefname{appendix}{Appendix}%
\makeautorefname{chapter}{Chapter}%
\makeautorefname{hy}{Hypothesis}%
\makeautorefname{section}{Section}%
\makeautorefname{subsection}{Section}%
\makeautorefname{subsubsection}{Section}%
\makeautorefname{theorem}{Theorem}%
\makeautorefname{thm}{Theorem}%
\makeautorefname{cor}{Corollary}%
\makeautorefname{lem}{Lemma}%
\makeautorefname{prop}{Proposition}%
\makeautorefname{pro}{Property}
\makeautorefname{conj}{Conjecture}%
\makeautorefname{defn}{Definition}%
% \makeautorefname{notn}{Notation}
% \makeautorefname{notns}{Notations}
\makeautorefname{rem}{Remark}%
\makeautorefname{quest}{Question}%
\makeautorefname{exmp}{Example}%
% \makeautorefname{ax}{Axiom}%
\makeautorefname{claim}{Claim}%
% \makeautorefname{ass}{Assumption}%
% \makeautorefname{asss}{Assumptions}%
% \makeautorefname{con}{Construction}%
% \makeautorefname{prob}{Problem}%
\makeautorefname{warn}{Warning}%
\makeautorefname{obs}{Observation}%
% \makeautorefname{conv}{Convention}%
%
%                  *** End of hyperref stuff ***

\theoremstyle{plain}
\newtheorem{thm}{Theorem}[section]
\newtheorem{cor}{Corollary}[section]
\newtheorem{prop}{Proposition}[section]
\newtheorem{lem}{Lemma}[section]
\newtheorem{prob}{Problem}[section]
\newtheorem{conj}{Conjecture}[section]
\newtheorem{claim}{Claim}[section]
%\newtheorem{ass}{Assumption}[section]
%\newtheorem{asses}{Assumptions}[section]

\theoremstyle{definition}
\newtheorem{defn}{Definition}[section]
% \newtheorem{ass}{Assumption}[section]
% \newtheorem{asss}{Assumptions}[section]
% \newtheorem{ax}{Axiom}[section]
% \newtheorem{con}{Construction}[section]
\newtheorem{exmp}{Example}[section]
% \newtheorem{notn}{Notation}[section]
\newtheorem{notns}{Notations}[section]
\newtheorem{pro}{Property}[section]
\newtheorem{quest}{Question}[section]
\newtheorem{rem}{Remark}[section]
\newtheorem{warn}{Warning}[section]
% \newtheorem{sch}{Scholium}[section]
\newtheorem{obs}{Observation}[section]
% \newtheorem{conv}{Convention}[section]

\newtheorem{exercise}{Exercise}[subsection]
% \newtheorem{myexercise}{Exercise}[subsection]
%
% \newenvironment{exercise}
%   {\begin{myexercise}}
%   {\par\vspace{-.5em}\noindent\hrulefill\vspace{.5em}\end{myexercise}}

%%%% hack to get fullref working correctly
\makeatletter
\let\c@obs=\c@thm
\let\c@cor=\c@thm
\let\c@prop=\c@thm
\let\c@lem=\c@thm
\let\c@prob=\c@thm
\let\c@con=\c@thm
\let\c@conj=\c@thm
\let\c@defn=\c@thm
\let\c@notn=\c@thm
\let\c@notns=\c@thm
\let\c@exmp=\c@thm
\let\c@ax=\c@thm
\let\c@pro=\c@thm
\let\c@ass=\c@thm
\let\c@warn=\c@thm
\let\c@rem=\c@thm
\let\c@sch=\c@thm
\let\c@hy=\c@thm
\let\c@equation\c@thm
\numberwithin{equation}{section}
\makeatother


\newcommand{\Gr}{\text{Gr}}
\DeclareMathOperator{\grad}{grad}
\DeclareMathOperator{\tr}{tr}
\DeclareMathOperator{\Wedge}{\bigwedge}
\DeclareMathOperator{\Res}{Res}
\DeclareMathOperator{\sgn}{sgn}
\DeclareMathOperator*{\rank}{rank}
\DeclareMathOperator{\height}{height}
\DeclareMathOperator{\Ann}{Ann}
\DeclareMathOperator{\Tor}{Tor}
\DeclareMathOperator{\Ext}{Ext}

% Number Theory (NT)
\renewcommand{\char}{\text{char}}
\DeclareMathOperator{\Gal}{Gal}

% Groups
\renewcommand{\sl}{\mathfrak{sl}}
\newcommand{\SU}{\bm{SU}}
\newcommand{\U}{\text{\textbf{U}}}
\newcommand{\CPn}{\C P^n}
\newcommand{\CP}{\C P}
\newcommand{\RPn}{\R P^n}
\newcommand{\RP}{\R P}
\DeclareMathOperator{\Sym}{Sym}
\newcommand{\Z}{\bm{Z}}
\newcommand{\A}{\bm{A}}
\newcommand{\F}{\bm{F}}
\newcommand{\Q}{\bm{Q}}
\newcommand{\R}{\bm{R}}
\newcommand{\C}{\bm{C}}
\newcommand{\N}{\bm{N}}
\renewcommand{\P}{\bm{P}}
\newcommand{\SL}{\text{\textbf{SL}}}
\newcommand{\GL}{\text{GL}}
\renewcommand{\S}{\mathbb{S}}
\DeclareMathOperator{\PGL}{PGL}
\DeclareMathOperator{\Lie}{Lie}

% Commutative Algebra
\DeclareMathOperator{\Frac}{Frac}

% Alg Geo (AG)
\DeclareMathOperator{\codim}{codim}
\DeclareMathOperator{\Spec}{Spec}
\DeclareMathOperator{\Specm}{Specm}
\DeclareMathOperator{\Proj}{Proj}
\DeclareMathOperator{\Jac}{Jac}
\DeclareMathOperator{\Prin}{Prin}
\DeclareMathOperator{\Div}{Div}
\DeclareMathOperator{\Supp}{Supp}
\DeclareMathOperator{\res}{res}
\DeclareMathOperator{\ord}{ord}

% Categories
\newcommand{\Ring}{\text{\textbf{Ring}}}
\newcommand{\Sch}{\text{\textbf{Sch}}}
\newcommand{\Top}{\text{\textbf{Top}}}
\newcommand{\Var}{\text{\textbf{Var}}}

\makeatletter
\newcommand{\colim@}[2]{%
  \vtop{\m@th\ialign{##\cr
    \hfil$#1\operator@font colim$\hfil\cr
    \noalign{\nointerlineskip\kern1.5\ex@}#2\cr
    \noalign{\nointerlineskip\kern-\ex@}\cr}}%
}
\newcommand{\colim}{%
  \mathop{\mathpalette\colim@{\rightarrowfill@\scriptscriptstyle}}\nmlimits@
}
\renewcommand{\varprojlim}{%
  \mathop{\mathpalette\varlim@{\leftarrowfill@\scriptscriptstyle}}\nmlimits@
}
\renewcommand{\varinjlim}{%
  \mathop{\mathpalette\varlim@{\rightarrowfill@\scriptscriptstyle}}\nmlimits@
}
\DeclareMathOperator{\End}{End}
\DeclareMathOperator{\coker}{coker}
\DeclareMathOperator{\Aut}{Aut}
\DeclareMathOperator{\Mor}{Mor}
\DeclareMathOperator{\Hom}{Hom}
\DeclareMathOperator{\im}{im}
\DeclareMathOperator{\id}{id}

% Algebraic Topology AlgTop
\DeclareMathOperator{\HH}{H}
